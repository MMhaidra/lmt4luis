%\documentclass[]{article}
\documentclass[twocolumn]{article}
\usepackage{amsmath,amsfonts}
%\usepackage{showlabels}
\usepackage[pdftex]{rotating,color}
%\usepackage[pdftex]{graphicx,color} 
\newcommand{\normal}[2]{{\cal N}(#1,#2)}
\newcommand{\NormalE}[3]{{\mathcal{N}}\left.\left(#1,#2\right)\right|_{#3}}
\renewcommand{\th}{^{\text{th}}}
\newcommand{\Radon}{{\cal R}}
\newcommand{\field}[1]{\mathbb{#1}}
\newcommand{\EV}[2]{\field{E}_{#1}\left[#2\right]}
\newcommand{\argmin}{\operatorname*{argmin}}
\newcommand{\argmax}{\operatorname*{argmax}}
\newcommand{\arrowmap}{\operatorname*{\rightarrow}}
\newcommand{\av}{\mathbf{v}}%
\newcommand{\bv}{\mathbf{w}}%
\newcommand{\Rv}{\mathbf{R}}%
\newcommand{\power}{p}
\newcommand{\like}{g}
\newcommand{\loglike}{L}
\newcommand{\logAP}{{\cal L}}
\newcommand{\z}{z}% The integrated rate, like a partition function
\newcommand{\area}{A} %The area or volume of a detector bin
\title{List Mode Tomography from Muon Shadows}

\author{Andy Fraser}
\begin{document}
\maketitle
\begin{abstract}
  We describe a procedure for estimating the difference between a
  nominal density profile and an unknown target density profile. The
  procedure uses detected cosmic ray muons that have passed through
  the unknown density as its source of information.  It uses
  background runs to characterize relevant features of the incident
  flux, detector efficiency and the nominal profile simultaneously.
\end{abstract}

\section{Introduction}
\label{sec:introduction}

In a classic study\cite{Alvarez70} Luis Alvarez et al.\ used
measurements of cosmic ray muons to search for hidden chambers in
Chephren's pyramid at Giza.  The study estimated the dependence of
muon flux on \emph{two} angular dimensions by counting the number of
detections in each $3\times 3$ degree bin in a square that was 90
degrees on each side.  On the basis of the estimated flux distribution
the study concluded that no hidden chambers exist.  We have built a
detector named \emph{LUIS}\cite{Green10} that characterizes each
measured muon in terms of \emph{four} parameters (two position
dimensions and two angular dimensions) that allows us to estimate
\emph{three} dimensional density profiles.  We describe the
tomographic algorithm for computing such estimates below.

The basis of list mode tomography is the following formula for the
unnormalized\footnote{Given a rate field $R$ with integral $z$, the
  discrete probability of the number of hits being the integer $M$ in
  a time interval $t$ is
  \begin{equation*}
    P(M|R,t) = \frac{e^{-zt} (zt)^M}{M!},
  \end{equation*}
  and given that value of $M$, the probability density for the hits
  falling at the locations $\gamma_1^M$ is
  \begin{equation*}
    f(\gamma_1^M|R,M) = z^{-M} \prod_{k=1}^M R(\gamma_k).
  \end{equation*}
  The product yields
  \begin{equation*}
    P(M,\gamma_1^M|R,t) = \frac{e^{-\z t}}{M!} \prod_{k=1}^M  R(\gamma_k),
  \end{equation*}
  the joint conditional probability mass and density function for
  $M$ hits with values $\gamma_1^M$.} likelihood of $M$ \emph{hits}
with values $\gamma_1^M$, drawn from an inhomogeneous Poisson process
\begin{equation}
  \label{eq:ihPoisson}
  \like(\gamma_1^M|R,t) = e^{-\z t} \prod_{k=1}^M  R(\gamma_k).
\end{equation}
I've used the following notation in Eqn.~\ref{eq:ihPoisson}
\begin{description}
\item[$\like$] The likelihood function
\item[$\gamma_k$] The coordinates of the $k^{\text{th}}$ measured
  hit.  For the application in this paper, $\gamma_k$ describes the
  trajectory of a detected muon.
\item[$\gamma_1^M=(\gamma_1,\gamma_2,\ldots,\gamma_M)$] The collection
  of all of the measured hits
\item[$R$] The field of detection rates
\item[$t$] The length of the time interval over which the data was
  taken
\item[$\z$] The integral of the rate field
\end{description}

If the detected trajectories are unique, then a rate field that is
zero everywhere except over volumes of size $\epsilon$ centered at the
points $\gamma_1^M$ where $R(\gamma)=\frac{1}{M\epsilon}$ has a
likelihood that is unbounded as $\epsilon \rightarrow 0$.  To avoid
such singularities, we will impose a prior on the rate fields.

For the task at hand, $R$ depends on a combination of the distribution
of incident muons at the earth's surface, the density of material
above the detector and the detector efficiency for each possible
trajectory $\gamma$.  We schematically summarize my forward model of
how the rate field $R$ depends on the density field $\rho$ by
\begin{equation}
  \label{eq:schematic}
  \rho(\xi) \arrowmap_{\Radon} u(\gamma)
  \arrowmap_{\Phi} R(\gamma),
\end{equation}
where the linear Radon transform $\Radon$ maps from the density field
in ordinary three dimensional space $\xi$ to a field $u$ in the four
dimensional space of possible trajectories $\gamma$, and the nonlinear
(but local in $\gamma$) function $\Phi$ depends on the energy
distribution of the incident muon flux.

We use a vector of parameters denoted by $\av$ to approximate the
density profile $\rho$ with a finite number of dimensions.  Using
$\av$ in place of $\rho$ in Eqn.~\ref{eq:schematic}, we write the log
likelihood as
\begin{equation*}
  \loglike(\gamma_1^M,\av) = -tz + \sum_{k=1}^M \log(R(\gamma_k|\av),
\end{equation*}
where $R(\gamma|\av)$ is the value of the rate field at $\gamma$, and
\begin{equation}
  \label{eq:z} 
  z = \int R(\gamma|\av) d\gamma 
\end{equation}
is the integrated rate.  In principal, we can finish the task by
selecting a prior $P(\av)$ for the density and solving for the maximum
a posteriori probability (MAP) $\av$, ie,
\begin{equation}
  \label{eq:vhat}
  \hat \av = \argmax_\av  \left( \loglike(\gamma_1^M,\av) +
    \log(P(\av)) \right).
\end{equation}

Subsequent sections explain details of an implementation of the
program indicated by \eqref{eq:vhat} and its application to
both simulations and measured data.

\section{Muon Flux and Interactions}
\label{sec:model}

We let $f(T,\gamma)$ denote the flux\footnote{At sea level, the total
  flux of cosmic ray muons is about one per $(\text{cm})^2$ per
  minute.} of muons at the earth's surface.  In this expression $T$ is
the energy and $\gamma\equiv(x,y,\theta,\phi)$ describes an incident
trajectory in terms of $(x,y)$ the intersection of the trajectory with
a horizontal plane at elevation $h=0$, $\theta$, the angle between the
path and a vertical line, and $\phi$ the angle between a vertical
projection of the trajectory and the $x$ axis.  We assume that the flux
factors into an amplitude $\tilde \alpha$ that depends on $\gamma$ and
a shape $F$ that depends on energy as follows
\begin{equation*}
  \tilde \Phi(T,\gamma) = \tilde \alpha(\gamma) \int_T^\infty
  f(\tau|\gamma) \equiv \tilde \alpha(\gamma) F(T).
\end{equation*}
The code that we have implemented can use either of the following forms
for the shape:
\begin{subequations}
  \label{eq:F}
  \begin{align}
    \label{eq:expF}
    & \text{exponential } & F(T) &=  e^{-\lambda T} \\
    \label{eq:PLF}
    & \text{power law } & F(T) &=
    \begin{cases}
      T^{1-\power} & T > 1.0 \\
      1 &  T \leq 1.0
    \end{cases}
  \end{align}
\end{subequations}

We model absorption by assuming that energy loss rate per distance is a
material property $\rho$ that does not depend on energy.  Since a ray
with trajectory $\gamma$ is absorbed if the integrated energy loss
along the trajectory exceeds the initial particle energy $T$, the
detection probability is
\begin{equation*}
  d(\gamma,T) =
  \begin{cases}
    \beta(\gamma) & T > u_\gamma \\
    0 & T \leq  u_\gamma,
  \end{cases}
\end{equation*}
where $\beta(\gamma)$ is the detector efficiency.  We assume that a
detector measures $\gamma$, both the position and angle, of muons that
penetrate the probed material.  We also assume that each $\gamma$
defines an entire trajectory, ie, that the material being probed slows
the speed of a muon but does not perturb its direction of motion.  In
summary, we write the detected particle flux in terms of the true
particle flux by replacing $\tilde \alpha(\gamma)$ with
$\alpha(\gamma)=\beta(\gamma)\tilde \alpha(\gamma)$, obtaining
\begin{equation*}
  \Phi(T,\gamma) = \beta(\gamma)\tilde\Phi(T,\gamma) =
  \alpha(\gamma)F(T).
\end{equation*}


\section{Maps Between Densities, Integrals, and Rates}
\label{sec:maps}

We restrict the task to estimating the density profile in a volume $V$
directly above the detector to a height $H$.  At that height we assume
that $\tilde \alpha(\gamma)F(T)$ describes the muon flux.  For an
incident muon with a trajectory defined by $\gamma =
(x,y,\theta,\phi)$, the line integral of the density along the
trajectory,
\begin{equation}
  \label{eq:u}
  u(\gamma) = \int_{h:\xi(h) \in V} \rho(\xi(h))  \frac{d s}{d h} dh
\end{equation}
where
\begin{align*}
  \xi(h) &= \begin{bmatrix}
    x + h \tan(\theta) \cos(\phi) \\
    y + h \tan(\theta) \sin(\phi) \\
    h
  \end{bmatrix} \\
  \frac{d s}{d h} &= \sqrt{1+(\tan(\theta))^2},
\end{align*}
combined with the detector efficiency $\beta(\gamma)$ determines the
probability of detection.  The field of integrals
\begin{equation*}
  u = \Radon \rho
\end{equation*}
is the \emph{Radon transform} of the density function $\rho$.

Combining the transformed field $u$ and an incident flux-efficiency
function $\Phi$, defines a field of \emph{rates} $R(\gamma)$ that
determines the probability of measuring a muon as a function of
position, angle, and duration with the probability density of a
detection at $\gamma_0$ given by
\begin{equation*}
  \lim_{\Delta \rightarrow 0} \frac{\text{Prob}(d=1|\gamma_0 \leq x
    \leq \gamma_0 + \Delta \gamma,0 \leq t \leq \Delta t)}{ \Delta
    \gamma \Delta t }.
\end{equation*}
Equivalently
\begin{equation*}
  R(\gamma) = \Phi(u(\gamma,\rho),\gamma)
\end{equation*}
where $\Phi$ maps the field of integrals $u$ to the field of rates
$R$.

To implement a numerical method, we choose a finite dimensional
representation of possible density fields $\rho$ by using coefficients
of a set of Gaussian basis functions with centers $\{\xi_k\}$ and
widths $\sigma$, ie,
\begin{align*}
  \rho(\xi) &= \sum_k \av_k \psi_{k}(\xi) + \rho_0(\xi)\\
  \psi_{k}(\xi) &= \left(2\pi \sigma^2\right)^{-\frac{3}{2}} \exp\left(
    -\frac{\left| \xi - \xi_k \right|^2}{2\sigma^2} \right),
\end{align*}
where $\rho_0$ is the nominal density.  For a trajectory
$\gamma=(x,y,\theta,\phi)$, the corresponding component of the Radon
transform of the density is
\begin{equation*}
  u(\gamma) = \Radon(\gamma) \av
\end{equation*}
where $\Radon(\gamma)$ is a functional defined as follows
\begin{align*}
  \Radon(\gamma)\av &\equiv \sum_k \Radon(\gamma)_k \av_k + \int_0^H 
  \rho_0\left( \xi(h) \right) \frac{ds}{dh} dh\\
  \Radon(\gamma)_k &\equiv \int_0^H \psi_k \left( \xi(h) \right) \frac{ds}
  {dh} dh,
\end{align*}
and $\xi$ and $\frac{ds} {dh}$ were defined for \eqref{eq:u}.  Note
that the operator $\Phi$ is local in $\gamma$ and the value of the
rate field at a trajectory $\gamma$ is
\begin{equation*}
  R(\gamma) =\Phi \left( \Radon(\gamma)\av, \gamma \right).
\end{equation*}

In summary, for a given vector of coefficients $\av$:
\begin{align}
  \rho(\xi) &= \sum_k \av_k \psi_{k}(\xi) \\
  u(\gamma) &= \Radon(\gamma) \av \\
  R(\gamma) &= \Phi (u(\gamma),\gamma) =
  \alpha(\gamma)F(u(\gamma)).
\end{align}

\section{An Objective Function}
\label{sec:LL}

To implement the procedure suggested by Eqn.~\eqref{eq:vhat}, we have
written code that implements the objective function (the log a
posteriori probability) and its first two derivatives.  Following
\eqref{eq:ihPoisson} we find that the log likelihood is
\begin{equation*}
  \loglike(\gamma_1^M, \av) = -tz + \sum_k \log \left( R(\gamma_k)
  \right).
\end{equation*}

\subsection{Using Background Data}
\label{sec:background}

Rather than trying to do the integral required to evaluate $z$
analytically, we estimate it using flat field or background data.
The following sum using small volumes $A_i$ to cover the domain of
$\gamma$ approximates the integral in \eqref{eq:z}
\begin{equation*}
  z(\av) \approx \sum_i \Phi(u(\gamma_i),\gamma_i) A_i.
\end{equation*}
To use Monte Carlo integration with importance sampling, one could
take $N$ draws from an iid random process with probability $P_i$ of
falling in $A_i$ and write
\begin{equation*}
   z(\av) \approx \sum_i \Phi(u(\gamma_i),\gamma_i)\frac{A_i}{P_i}
   \frac{N_i}{N}.
\end{equation*}
If instead of having $N$ draws from an iid process, one had Poisson
processes with parameter $\lambda_i$ for each volume $A_i$ one could
use the approximation
\begin{equation*}
   z(\av) \approx \sum_i \Phi(u(\gamma_i),\gamma_i)\frac{A_i N_i}
   {\lambda_i}.
\end{equation*}
Data taken over a time interval $t_b$ with a known background
density\footnote{If $\rho_0$ completely describes the background
  density, then $\av_b = 0$.}  $\av_b$ can provide the Poisson
processes.  In the limit of small volumes
\begin{align*}
  \lim_{A_i\rightarrow 0} \frac{A_i}{\lambda_i} =
  \frac{1}{t_b \Phi(\Radon(\gamma_i)\av_b,\gamma_i)},
\end{align*}
where we have written the Radon transform of the background density as
$\Radon(\gamma_i)\av_b$.  Substitution yields
\begin{equation*}
  \hat z(\av) = \sum_j \frac{\Phi(\Radon(\gamma_j)\av,\gamma_j)}
  {t_b\Phi(\Radon(\gamma_j)\av_b,\gamma_j)}
\end{equation*}
as an estimate of $z$.  If $\Phi$ factors into an amplitude $\alpha$
and shape $F$, the amplitudes cancel and
\begin{equation}
  \label{eq:zhat}
  \hat z(\av) = \frac{1}{t_b}\sum_j \frac{F(\Radon(\gamma_j)\av)}
  {F(\Radon(\gamma_j)\av_b)}.  
\end{equation}
Using \eqref{eq:zhat} yields the following approximation for the
log likelihood
\begin{align*}
  &-\frac{t}{t_b}\sum_j \frac{F(\Radon(\gamma_j)\av)}
  {F(\Radon(\gamma_j)\av_b)} + \sum_k \log(F(\Radon(\gamma_k)\av)) \\
  &\quad + \sum_k \log(\alpha(\gamma_k)).
\end{align*}
Since the last term is independent of $\av$, henceforth we simply
use
\begin{align}
  \loglike(\gamma_1^M,\av|{\tilde y}_1^N,\av_b) & + C \approx \sum_{k=1}^M
  \log(F(\Radon(\gamma_k)\av)) \nonumber \\
  \label{eq:LL}
  & -\frac{t}{t_b}\sum_{j=1}^N \frac{F(\Radon(\gamma_j)\av)}
  {F(\Radon(\gamma_j)\av_b)},
\end{align}
where $j$ indexes the background data $\tilde \gamma_1^N$, $k$
indexes the target data $\gamma_1^M$ and $C$ is a constant that I
henceforth ignore.

\subsection{Regularization}
\label{sec:regularization}

While one could use \eqref{eq:LL} to search for a maximum likelihood
estimate of the coefficient vector $\av$, we subtract an approximate
\emph{total variation} term
\begin{equation*}
  T(\av) = \alpha \sqrt{\av^T D \av + \beta}
\end{equation*}
from the log likelihood.  We design $D$ so that $\av^T D \av$
approximates the total squared variation, ie, $\av^T D \av = \av^T
\tilde D^T \tilde D \av$, where $\tilde D$ approximates the gradient.
Thus $D \equiv \tilde D^T \tilde D$ is positive semidefinite by
construction.  Such regularization is equivalent to imposing a prior
$P(\av) \propto e^{-T(\av)}$ and then solving for the maximum a
posteriori probability $\av$.  The derivatives of $T$ are
\begin{align*}
  T'(\av) &= \alpha^2 \frac{\av^TD}{T(\av)} \\
  T''(\av) &= \alpha^2 \frac{T(\av)D - D\av T'(\av)}{(T(\av))^2} \\
  &= \alpha^2 \frac{(T(\av))^2D - \alpha^2D\av \av^T D}{(T(\av))^3}.
\end{align*}
Using the Cauchy Schwarz inequality, one can show that for any vector
$w$, $w^T(T'')w \geq \beta > 0$.  Thus the second derivative $T''$ is
positive definite.

\subsection{Derivatives}
\label{sec:derivatives}

Dropping constants, the log a posteriori probability and its first two
derivatives with respect to $\av$ are
\begin{align}
  \label{eq:LAP}
  \logAP &= \sum_{k=1}^M \log(F(u_k))
  -\sum_{j=1}^N \frac{t}{t_b} \frac{F(u_j)}
  {F(\Radon_j\av_b)} - T(\av) \\
  \label{eq:D1LAP}
  \logAP' &= \sum_k \frac{F'(u_k)}{F(u_k)}\Radon_k -\sum_j\frac{t}{t_b}
  \frac{F'(u_j)}{F(\Radon_j \av_b)} \Radon_j - T'(\av) \\
  \logAP'' &= \sum_k \Radon_k^T \frac{F(u_k)F''(u_k)-(F'(u_k))^2}
  {(F(u_k))^2} \Radon_k \nonumber \\
  \label{eq:D2LAP}
  &\quad -\sum_j\frac{t}{t_b} \Radon_j^T \frac{F''(u_j)}{F(\Radon_j
    \av_b)} \Radon_j - T''(\av),
\end{align}
where I've used the abbreviations $\Radon_i \equiv \Radon(\gamma_i)$
and $u_i \equiv \Radon_i \av$.

If $\av = \av_b$ then
\begin{equation*}
   \sum_k \Radon_k^T \frac{F''(u_k)} {F(u_k)} \Radon_k \approx
   \frac{t}{t_b}\sum_j \Radon_j^T \frac{F''(u_j)}{F(\Radon_j \av_b)}
   \Radon_j
\end{equation*}
and
\begin{equation*}
  \logAP'' \approx \sum_k \Radon_k^T \frac{-(F'(u_k))^2} {(F(u_k))^2} \Radon_k -
  T''(\av)
\end{equation*}
which is negative definite.  Thus we expect $\logAP''$ to be negative
definite over a set of $\av$ values in a neighborhood of $\av_b$, and
in that set, $\logAP$ is concave and has a single local/global
maximum.  With effective algorithms to evaluate the expressions for
$\logAP$, $\logAP'$, and $\logAP''$, one can use one of many different
optimization algorithms to find the value $\hat \av $ that maximizes
$\logAP((\gamma_1^M,\av)$.

\section{Simulation}
\label{sec:simulation}

Figure~\ref{fig:sim1} illustrates the result of applying the
optimization suggested by Eqn.~\eqref{eq:vhat} to simulated muon
measurements.  We used the python/scipy/optimize/fmin\_ncg\cite{scipy}
function that implements the Newton conjugate gradient
algorithm.  This section describes the simulation and the details of
my implementation\footnote{Code that produced the figures in this
  paper is available at FixMe.}.

\marginpar{Estimate variance}
\begin{figure*}
  \centering 
  \resizebox{0.32\textwidth}{!}{\input sim1true.pdf_t } 
  \resizebox{0.32\textwidth}{!}{\input sim1hat.pdf_t }
  \resizebox{0.32\textwidth}{!}{\input sim2hat.pdf_t }  
  \caption{Reconstruction of a density profile from simulated muon shadows.}
  \label{fig:sim1}
\end{figure*}

The detector is a rectangle with dimensions $D_x$, $D_y$ located in
the $h=0$ plane.  The centers of the basis functions are arranged in a
\emph{hexagonal close packed} pattern\footnote{See
  http://en.wikipedia.org/wiki/Close-packing} inside a rectangular box
centered above the detector that has dimensions $D_x$, $D_y$, $D_h$.
The distance from the bottom face of the box to the detector is $h_0$.
The minimum center to center spacing for the basis functions is $d_0$,
and the width of the basis function is $\sigma = \frac{d_0}{2}$.  We
truncate the basis functions for distances beyond
$\left|x-\xi_k\right| > \sqrt{3.5}\sigma$.  The nominal density is
zero everywhere except for a unit density slab between the elevations
$h_0$ and $h_0+D_h$.  Thus at $\gamma_j \equiv
(x_j,y_j,\theta_j,\phi_j)$ the value of the Radon transform of the
nominal density is
\begin{equation*}
  \Radon_j \av_b = \frac{D_h}{\cos(\theta_j)}.
\end{equation*}
The target density differs from the nominal density in a spherical
region of radius $r_s$ centered at $(x_s,x_y,x_z)$.  In that sphere,
the density is $1+\Delta_\rho$.

\subsection{Total Variation}
\label{sec:Variation}

We build the matrix $D$ by adding a $4 \times 4$ block that is
approximately proportional to the squared gradient at the center of
each unit tetrahedron in the lattice of basis functions.  The vertices
of a unit tetrahedron are the centers of 4 basis functions.  If the
coefficients of those basis functions are $(v_a,v_b,v_c,v_d)$ then
the contribution of that tetrahedron to the quadratic form is
\begin{equation*}
  \begin{bmatrix} v_a&v_b&v_c&v_d \end{bmatrix} D 
  \begin{bmatrix} v_a \\ v_b \\ v_c \\ v_d \end{bmatrix}  
\end{equation*}
where\footnote{See http://en.wikipedia.org/wiki/Tetrahedron for an
  explanation of $\tilde D$.}
\begin{align*} \\
  D &= \tilde D^T \tilde D \\
  \tilde D &=
  \begin{bmatrix}
    1 & -1& -1& 1 \\
    1 & 1-& 1 & -1 \\
    1 & 1 & -1& -1
  \end{bmatrix}\\
  D &=
  \begin{bmatrix}
    3 & -1 & -1 & -1 \\ -1 & 3 & -1 & -1 \\ -1 & -1 & 3 & -1 \\
    -1 & -1 & -1 & 3
  \end{bmatrix}.
\end{align*}

\subsection{Caching $\Radon_k$ Vectors on Disk}
\label{sec:cache}

So that the code can handle millions of detected particles, for each
detection, $\gamma_k$, we store $\Radon_k$ on disk.  Since the fraction
of nonzero entries in $\Radon_k$ is of order\footnote{The following
  calculations suggest that number of nonzero entries in $\Radon_k$ is
  about $\frac{s}{2d_0}$ where $s$ is the length of the trajectory
  though the test volume and $\sigma=\frac{1}{2}$ is the range of the
  basis functions as a fraction of $d_0$.
  \begin{align*}
    &\text{volume of cylinder} && s \pi 3.5 (d_0)^2 \sigma^2 \\
    &\text{sphere packing fraction} && \frac{\pi}{3\sqrt{2}} \\
    &\text{volume of sphere} && \frac{4\pi (d_0)^3}{3} \\
    &\text{volume per center} && 4\sqrt{2}(d_0)^3 \\
    &\text{number of centers in cylinder} && \frac{s}{d_0} \frac{3.5
      \pi \sigma^2}{4\sqrt{2}} \approx 0.496 \frac{s}{d_0}
  \end{align*}
}
$\frac{(d_0)^2\sigma^2}
{D_xD_y}$, we use a sparse format to represent each $\Radon_k$.  For
each $k$, we write the following data sequence on disk:
\begin{description}
\item[$n$] The number of indices that follow (also the number of values)
\item[$i_1^n$] The $n$ integers that identify the components of
  $\av$ associated with the values to follow
\item[$r_1^n$] The $n$ nonzero entries in $\Radon_k$
\item[$\theta_k$] The azimuthal angle
\item[$n_k$] The number of hits with trajectory parameters
  $\gamma_k$.  Although in true list mode $n_k = 1$, for efficiency, I
  may finely quantize the trajectory measurements.
\end{description}
As we calculate $\Radon_k$ and append the above data to a disk file, I
keep track of the maximum value of $n$ so that we can allocate and
reuse memory for a sparse matrix of that size to do subsequent
calculations.

\subsection{Estimating the Variance}
\label{sec:variance}

The a posteriori distribution of the density is approximately the
Gaussian $\normal{\hat \av}{\hat \Sigma}$ with
\begin{equation*}
  \hat \Sigma^{-1} = \logAP''.
\end{equation*}
Even if the problem size makes inverting $\hat \Sigma^{-1}$ difficult, one
may extract elements of $\hat \Sigma$ to characterize the uncertainty of
the density estimate as follows.  Note that
\begin{equation*}
  \left( \hat \Sigma \right)_{i,j} = w_i^T \hat \Sigma w_j
\end{equation*}
where $w_i$ and $w_j$ are unit vectors.  If $x = \hat \Sigma w_j$, then
$x$ solves
\begin{equation*}
  \hat \Sigma^{-1} x = w_j.
\end{equation*}
Since we can operate with $\hat \Sigma^{-1}$ without even forming the
whole matrix, we can use a conjugate gradient method to solve for $x$
and the calculate
\begin{equation*}
  \left( \hat \Sigma \right)_{i,j} = w_i^T  x.
\end{equation*}

\section{Physical Example}
\label{sec:physical}

Figure~\ref{fig:exp1} is a slice of a density profile estimated from
data taken with the LANL LUIS detector.  This section describes that
detector, the data\footnote{The data consist of 4 files in \emph{root}
  format.  Since each file is about 21 Gigabytes the total exceeds the
  limit for inclusion in the supplementary material.  However, we do
  make it available at FixMe.}, and the modifications that we made to
the simulation code before applying it to the data.

\begin{figure}
  \centering
  \includegraphics[angle=-90,width=0.4\textwidth]{LUISbricks_5.jpg}
  \resizebox{0.48\textwidth}{!}{\input exp1hat.pdf_t }  
  \caption{Density profile reconstructed from measured muon shadows.}
  \label{fig:exp1}
\end{figure}

Andrew Green's data come in segments that were measured with the
instrument at different positions.  We'll use the following notation to
describe that data and the algorithms we apply to it:
\begin{description}
\item[$I$] Index of segment
\item[$k$] Index of a single muon measurement
\item[$K_I$] The set of measurements in segment $I$
\item[$I(k)$] The segment that $k$ is in, ie,  $I(k) = I:k\in K_I$
\item[$t_I$] The duration of segment $I$
\item[$t_k = t_{I(k)}$] The duration of the segment of which $k$ is a member
\item[$t_{b(k)} = \sum_{I\neq I(k)} t_I$] The duration of the back
  ground data for $k$
\item[${\cal J} = \{(I,k):I\neq I(k)\}$] The set of all segment
  trajectory pairs excluding the paring of any trajectory with the
  segment of which it is a member
\item[$\Gamma_I$] Trajectory offset for segment $I$
\item[$\gamma_k$] Trajectory of measurement $k$ in the instrument frame
\item[$\Radon_\gamma$] Radon vector for trajectory $\gamma$
\item[$\tilde u_{I,k}$] The $u$ field at $\gamma = \Gamma_I +
  \gamma_k$ for the empty or background density
\item[$u_{I,k}= \Radon_{\Gamma_I + \gamma_k} \av + \tilde u_{I,k}$]
  The scalar value of $u(\av)$ at $\gamma = \Gamma_I + \gamma_k$
\end{description}

The generalization of the likelihood for multiple segments is
\begin{equation*}
  \like(\gamma_1^M|R) = \prod_I e^{-\z_I t_I} \prod_{k \in K_I}  t_I R(u_{I,k}).
\end{equation*}
For each segment $I$, using data from other segments as background to
estimate $z$ leads to the following approximation for the log likelihood
\begin{equation*}
  \small L = \sum_I \left[ \sum_{k\in K_I} \log \left( t_I F(u_{I,k})
    \right) - \sum_{J\neq I} \sum_{k\in K_J} \frac {t_I}{t_{b(k)}}
    \frac{F(u_{I,k})}{F(\tilde u_{I,k})} \right]
\end{equation*}
or equivalently
\begin{equation}
  \label{eq:PELL}
  L = \sum_k \log(t_k F(u_{I(k),k})) - \sum_{(I,k) \in {\cal J}} \frac{t_{I}}{t_{b(k)}}
  \frac{F(u_{I,k})}{F(\tilde u_{I,k})}.
\end{equation}
Since the two terms in Eqn.~\eqref{eq:PELL} have the same form as the
first two terms of Eqn.~\eqref{eq:LAP}, we can use the same code to
solve the optimization problem by first calculating the offset
trajectories $\Gamma_I+\gamma_k$ and caching the corresponding Radon
functionals $\Radon_{\Gamma_I+\gamma_k}$.

\section{To Do}
\label{sec:todo}

\begin{enumerate}
\item Describe binning of experimental data
\item Acknowledge Schwitters and UT
\item I don't know the meaning of the scale of my estimated density
  for the real data
\end{enumerate}
\bibliographystyle{plain}
\bibliography{small}

\end{document}


%%% Local Variables:
%%% eval: (TeX-PDF-mode)
%%% End:
