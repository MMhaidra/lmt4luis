%\documentclass[]{article}
\documentclass[twocolumn]{article}
\usepackage{amsmath,amsfonts}
%\usepackage{showlabels}
\usepackage[pdftex]{rotating,color}
%\usepackage[pdftex]{graphicx,color} 
\newcommand{\normal}[2]{{\cal N}(#1,#2)}
\newcommand{\NormalE}[3]{{\mathcal{N}}\left.\left(#1,#2\right)\right|_{#3}}
\renewcommand{\th}{^{\text{th}}}
\newcommand{\Radon}{{\cal R}}
\newcommand{\field}[1]{\mathbb{#1}}
\newcommand{\EV}[2]{\field{E}_{#1}\left[#2\right]}
\newcommand{\argmin}{\operatorname*{argmin}}
\newcommand{\argmax}{\operatorname*{argmax}}
\newcommand{\arrowmap}{\operatorname*{\rightarrow}}
\newcommand{\av}{\mathbf{v}}%
\newcommand{\bv}{\mathbf{w}}%
\newcommand{\Rv}{\mathbf{R}}%
\newcommand{\power}{p}
\newcommand{\like}{g}
\newcommand{\loglike}{L}
\newcommand{\logAP}{{\cal L}}
\newcommand{\z}{z}% The integrated rate, like a partition function
\newcommand{\area}{A} %The area or volume of a detector bin
\title{Supplementary Guide for LMT4LUIS\footnote{LA-CC-09-094}}

\author{Andy Fraser}
\begin{document}
\maketitle
\begin{abstract}
  Here I provide a guide to material that supplements the article
  \emph{List Mode Tomography from Muon Shadows}.
\end{abstract}

\section{Introduction}
\label{sec:introduction}

For now, I include the same figures that appear in LMTMT.  Later, I
will use different figures and different text.

\begin{figure*}
  \centering 
  \resizebox{0.32\textwidth}{!}{\input TestVA1M.pdf_t } 
  \resizebox{0.32\textwidth}{!}{\input TestVB1M.pdf_t }
  \resizebox{0.32\textwidth}{!}{\input SimVB.pdf_t }  
  \caption{Reconstruction of a density profile from simulated muon shadows.}
  \label{fig:sim1}
\end{figure*}

\begin{figure}
  \centering
  \includegraphics[angle=-90,width=0.4\textwidth]{LUISbricks_5.jpg}
  \resizebox{0.48\textwidth}{!}{\input JAGbrick.pdf_t }  
  \caption{Density profile reconstructed from measured muon shadows.}
  \label{fig:exp1}
\end{figure}

In the paper, I wrote the regularization as
\begin{equation*}
  T(\av) = \alpha \sqrt{\av^T D \av + \beta}.
\end{equation*}
In the code, I use
\begin{equation*}
  T(\av) = \alpha \sqrt{n\av^T D \av + \beta^2},
\end{equation*}
where $n$ is the dimension of $\av$.  I use $n\av^T D \av$ because as
$n\rightarrow \infty$ that form converges.  With the second form, the
regularization is approximately independent of $n$ and
$\frac{T}{\alpha\beta}$ indicates how close the regularization is to
the absolute value $\alpha\left|n\av^T D \av\right|$.

\section{To Do}
\label{sec:todo}

\begin{enumerate}
\item Describe binning of experimental data
\item Talk about code
\item List prerequisites
\item List key routines
\item Talk about Tkinter
\item Put the disk caching stuff here
\end{enumerate}

\end{document}


%%% Local Variables:
%%% eval: (TeX-PDF-mode)
%%% End:
